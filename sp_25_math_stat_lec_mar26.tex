\documentclass{beamer}

\usepackage{amsmath, amssymb}
\usepackage{tikz-cd}
\usepackage{xcolor}
\usepackage{graphicx}

\title{MAT434: Theory of Mathematical Statistics}
\author{\textbf{Miraj Samarakkody}}
\institute{Tougaloo College}
\date{03/24/2025}

\begin{document}

\begin{frame}
    \titlepage
\end{frame}



\begin{frame}{}
    \begin{center}
        \Huge{2.3 Functions of a Random Variable}
    \end{center}

\end{frame}





\begin{frame}{Proposition A}
\begin{block}{Proposition A}    
If \(X \sim N(\mu, \sigma^2)\) and \(Y=aX+b\), then \(Y\sim N(a\mu +b, a^2\sigma^2)\)
\end{block}
\end{frame}

\begin{frame}{Applications of Proposition A}
Consider the random variable \[Z=\dfrac{X-\mu}{\sigma}\]\\ \pause 


\begin{itemize}
    \item We can see that \(Z\) follows a standard normal distribution. \\\pause


    \item Also we can see that probabilities for general normal random variables can be evaluated in terms of probabilities fro standard normal random variables. 
\end{itemize} 
\end{frame}




\begin{frame}{Example B}
Let \(X \sim N (\mu, \sigma^2)\), and find the probability that \(X\) is less than \(\sigma\) away from \(\mu\); that is, find \(P(|X-\mu|< \sigma)\).\\ \pause
\vspace{0.2in}

Thus, a normal random variable is within 1 standard deviation of its mean with probability \(0.68\)
\end{frame}

\begin{frame}{Example C}
    Find the density of \(X=Z^2\), where \(Z \sim N(0,1)\). \\
    \vspace{0.2in}
    
\end{frame}





\end{document}
\documentclass{beamer}

\usepackage{amsmath, amssymb}
\usepackage{tikz-cd}
\usepackage{xcolor}
\usepackage{graphicx}

\title{MAT434: Theory of Mathematical Statistics}
\author{\textbf{Miraj Samarakkody}}
\institute{Tougaloo College}
\date{02/12/2025}

\begin{document}

\begin{frame}
    \titlepage
\end{frame}

\begin{frame}
    Refer the operation table. \\ \pause
    What is closure property? \\ \pause
    Identity\\ \pause
    Inverse\\ \pause
    Commutative or Abelian

\end{frame}

\begin{frame}
    \begin{block}{The Dihedral Group}
        The dihedral group of order \(2n\) is often called the group of symmetries of a regular \(
        n\)-gon. 
    \end{block}
\end{frame}

\begin{frame}{Groups}
    Let \(G\) be a set \(A\) binary operation on \(G\) is a function that assigns each ordered pair of elements of \(G\) an element of \(G\). \pause

    \begin{block}{Definition}
        Let G be a set together with a binary operation that assigns to each ordered pair \((a,b)\) of elements of \(G\) an elemnts in \(G\) denoited by \(ab\). The \(G\) is a group if the following three properties are satisfied. \pause
        \begin{enumerate}
            \item \textbf{Associativity}: \((ab) c= a (bc)\) for all \(a,b \in G\).
            \item \textbf{Identity}: There is an element \(e\) (called identity) in \(G\) such that \(ae=ea=a\) for all \(a \in G\).
            \item \textbf{Inverses}: For each element \(a \in G\), there is an element \(b\in G\) such that \(ab=ba=e\).
        \end{enumerate}
        If for any \(a,b \in G\), \(ab=ba\) then group is an abelian group. 
    \end{block}
\end{frame}

\begin{frame}{Examples}
    \begin{enumerate}
        \item The set of integers \(\mathbb{Z}\), the set of rational numbers \(\mathbb{Q}\), and the set of real numbers \(\mathbb{R}\), are all groups under ordinary addition.\pause
        \item The subset \(\{1,-1, i, -i\}\) of the complex numbers under multiplication is a group. \pause
        \item The set \(S\) of positive irrational numbers together with 1 under multiplication is now a group.\pause
        \item A rectangular array of the form \(\begin{bmatrix}
            a & b\\
            c & d
        \end{bmatrix}\) is called a \(2\times 2\) matrix. The set of all \(2\times 2\) matrices with real entries is a group under componentwise addition. \[
        \begin{bmatrix}
            a_1 & b_1\\
            c_1 & d_1
        \end{bmatrix} + \begin{bmatrix}
            a_2 & b_2\\
            c_2 & d_2
        \end{bmatrix} = \begin{bmatrix}
            a_1 + a_2 & b_1 + b_2\\
            c_1 + c_2 & d_1 + d_2
        \end{bmatrix}
        \]
    \end{enumerate}
\end{frame}

\begin{frame}{Examples Cont.}
    \begin{enumerate}
        \setcounter{enumi}{4}
        \item The set \(\mathbb{Z}_n =\{0,1, \dots, n-1\}\) for \(n \geq 1\) is a group under addition modulo \(n\). This group is ususally reffered to as the group of integers modulo \(n\). \pause
        \item The set \(\mathbb{R}^*\) of non-zero real numbers is a group under ordinary multiplication. The identity is 1 and the inverse of \(a\) is \(1/a\). \pause
        \item The set of all invertible \(n\times n\) matrices with real entries is a group under matrix multiplication. This group is denoted by \(GL(n,\mathbb{R})\). 
        \[
        GL(2, \mathbb{R}) = \left\{ \begin{bmatrix}
            a & b\\
            c & d
        \end{bmatrix} \mid ad-bc \neq 0 \text{ for } a,b,c,d \in \mathbb{R}\right\}
        \]\pause
        What happens when \(ad-bc=0\)?
    \end{enumerate}
    
\end{frame}






\end{document}